\index{Meatloaf}\index{Meatloaf!Grandma's}
\begin{recipe}{Meatloaf}{Serves ?}{? minutes}
\freeform This is the meatloaf recipe that Linda made for the family while in Cleveland.  It is from Linda's grandmother.

\ing[2]{tbsp}{butter, unsalted, softened}
Butter a bundt pan and set aside.
\ing[2]{tbsp}{butter, unsalted, melted}
\ing[1]{c.}{milk, warm}
\ing[\fr13]{c.}{water, warm}
\ing[\fr14]{c.}{granulated sugar}
\ing[1]{pkg.}{rapid rise yeast}
In large measuring cup mix together.
\ing[3\fr14]{c.}{\apf{}}
\ing[2]{tsp}{salt}
Mix flour and salt in bowl of standing mixer fitted with dough hook.
\freeform
Turn machine to low and slowly add milk mixture.  After dough comes
together, increase speed to medium and mix until dough is shiny and
smooth, 6 to 7 minutes.Turn dough onto lightly floured surface and
knead briefly to form smooth, round ball.Coat large bowl with
oil. Place dough in bowl and coat surface of dough lightly with oil.
Cover bowl with a kitchen towel and place in warm oven until dough
doubles in size, 50 to 60 minutes.
\ing[1]{c.}{packed light brown sugar}
\ing[2]{tsp}{ground cinnamon}
\ing[8]{tbsp}{melted butter}
While the dough is rising, mix brown sugar and cinnamon together in bowl. Place melted butter in second bowl. Set aside.
\freeform
Gently remove dough from bowl and pat into rough 8-inch square. Using
bench scraper or knife, cut dough into 64 pieces. Roll each piece of
dough into a ball. Working a few at a time, dip balls in melted
butter, then roll in brown sugar mixture. Layer balls in Bundt pan
staggering seems where dough balls meet as you build layers. Cover
bundt pan tightly with plastic wrap and place in turned off oven until
dough balls are puffy and have risen 1 to 2 inches from the top of the
pan, 50 to 70 minutes. Remove pan from oven and heat oven to 350
degrees. Unwrap pan and bake until top is deep borwn and caramel
begins to bubble around edges, 30 to 35 minutes. Cool in pan for 5
minutes, then turn out onto platter and allow to cool slightly, about
10 minutes.
\ing[1]{c.}{powdered sugar}
\ing[2]{tbsp}{milk}
While the bread cools, whisk powdered sugar and milk in a small bowl until no lumps remain. Drizzle glaze over warm monkey bread and serve warm. 
\end{recipe}
%% \begin{figure}
%% \begin{center}
%% \includegraphics[width=0.8\textwidth]{\string~/Dropbox/cookbook/figures/monkey-bread}
%% %% \includegraphics[width=0.3\textwidth]{\string~/Dropbox/cookbook/figures/monkey-bread-2}
%% %% \hspace{0.1\textwidth}
%% %% \includegraphics[height=0.25\textwidth]{\string~/Dropbox/cookbook/figures/chemex-2}
%% \end{center}
%% \caption*{Monkey bread}
%% \end{figure}
