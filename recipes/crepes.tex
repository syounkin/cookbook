\begin{recipe}{Crepes}{\unit[]{}}{\unit[20]{minutes}}
\freeform Julie is not sure where this recipe came from. Julie and Annie like to eat crepes or Swedish pancakes with a thin layer of apricot jam. 
\ing[\fr34]{c.}{unbleached all-purpose flour}
\ing[\fr12]{tsp}{salt}
\ing[1]{tsp}{baking powder}
\ing[2]{tbsp}{powdered sugar}
Whisk together dry ingredients in a medium bowl
\ing[2]{}{eggs}
\ing[\fr23]{c.}{milk}
\ing[\fr13]{c.}{water}
\ing[\fr12]{tsp}{vanilla extract}
Whisk together wet ingredients. Whisk wet ingredients into dry ingredients.
\ing[]{}{butter}
\freeform Heat a nonstick frying pan to slightly hotter than medium heat and melt about 1 tablespoon of butter in the pan until it just starts to sizzle. Ladle about \fr14 to \fr13 cup of batter into the pan. (How much batter you use here depends on the size of your pan and the extent of your crepe-flipping skills.) Gently but quickly tilt the pan around in a circle to spread the batter out in a thin layer across the bottom of the frying pan. When golden brown, gently flip the crepe over and fry until golden. 
Every one or two crepes, put  a fresh tablespoon of butter in the pan before frying the next crepe.
Jam, maple butter, powdered sugar, cinnamon sugar and nutella all make excellent spreads atop these crepes.
\end{recipe}