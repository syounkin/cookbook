\index{Jambalaya!Roast Chicken}
\index{Stew!Jambalaya!Roast Chicken}
\begin{recipe}{Roast Chicken Jambalaya}
{\unit[6]{servings}}{\unit[45]{minutes}}
\freeform This recipe from Steve Younkin was adapted from one pulled
from the internet. The sausage makes this dish so find a sausage you
like in your jambalaya. Steve always uses Savoie andouille, and avoids
other brands. You can add almost any meat to jambalaya (e.g. shrimp
which will cook well if you add them raw after the rice is done, or
white meat chicken).
\ing[5]{oz}{Andouille sausage}
\ing[2-3]{tbsp}{olive oil}
Slice the sausage into quarter inch slices, then pile up 3 slices and
cut the stack into quarters. Repeat until all the sausage is in
quarter inch pieces. Add the sausage to olive oil in an enameled cast
iron pot over medium heat. Make sure the flat surface of each piece of
sausage is down on the pan. I prefer Savoie Andouille sausage.
\ing[1]{large}{white onion}
While the sausage is cooking, dice the onion into quarter inch pieces
and layer gently over the sausage so that the sausage continues to
brown.
\ing[5-8]{stalks}{celery}
Next dice the celery into quarter inch pieces, add to the pot, and mix
everything up getting any brown bits off the bottom of the pan.
\ing[1]{}{red pepper}
Next dice the red pepper into quarter inch pieces and add to the pot.
\ing[1]{tsp}{dried thyme}
\ing[1]{tsp}{dried oregano}
\ing[1-2]{tbsp}{dried basil}
Next add the basil, thyme, and oregano. Mix well and cook for 3-5
minutes to wilt the vegetables completely and release the oil from the
herbs.
\ing[2]{cups}{vegetable or chicken stock}
\ing[1]{cup}{white rice}
Add stock (I prefer Kitchen Basics) and white rice. Bring to a boil
and sprinkle salt lightly over the entire surface of the stock. Reduce
the heat to very low, cover and allow the rice to absorb the flavorful
stock for 15 minutes.
\ing{}{chicken, dark meat, roasted}
Cut the dark meat from an entire roasted chicken into bite
sized pieces and mix into the rice.\\
\freeform Test to make sure the rice is completely done (cook longer if
needed) then let the jambalaya sit for 5-10 minutes to completely warm
up the chicken. Serve with warm bread.
\end{recipe}
