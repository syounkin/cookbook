\index{Chicken!Crispy Chicken Filets}
\begin{recipe}{Crispy Chicken Filets}{\unit[6-8]{pieces}}{\unit[1]{hour}}
\freeform These filets make excellent crispy chicken sandwiches and have a
well-seasoned, crunchy exterior. They can be eaten hot or later the
same day at room temperature. Cut the breasts into longer strips to
make chicken fingers.
\newstep Heat to \unit[365\0]{F.} 3-4 inches of any combination of vegetable or
peanut oil in a heavy bottomed saucepan.
\ing[1]{}{large boneless, skinless chicken breast}
\ing[2]{}{egg whites}
Filet the chicken breast lengthwise, then cut each filet in half
widthwise to yield 4 pieces. Pound the chicken to \fr14 inch thickness
between two pieces of plastic wrap or butchers paper. Lightly beat the
egg whites and set aside in a wide shallow bowl.
\ing[1\fr12]{tsp}{kosher salt}
\ing[1]{tsp}{black pepper}
\ing[1]{tsp}{garlic powder}
\ing[\fr12]{tsp}{dried thyme}
\ing[\fr12]{tsp}{dried sage}
\ing[\fr12]{tsp}{cayenne pepper}
Mix together in a small bowl
\ing[1\fr12]{c.}{all purpose flour}
\ing[1]{tsp}{baking powder}
\ing[3]{tbsp}{water}
\ing[1]{tbsp}{above spice mixture}
Whisk together flour, baking powder and 1 tablespoon of the spice mixture (reserving the rest). Sprinkle the
water over top and with your fingers, rub the water into the flour
until shaggy pieces form.
\freeform Season the chicken breast with the remaining spice mixture. Dip each
piece of chicken in the egg whites, letting extra drip back into the
bowl, then press the chicken into the shaggy flour mixture, sprinkling
more over top and pressing into the chicken until fully and evenly
coated. Set on a wire rack over a cookie sheet. When each piece of
chicken has been seasoned, dipped and coated in flour, place the wire
rack in the refrigerator for 30 minutes to 2 hours. (This step is not
necessary but makes for an especially crispy coating when fried.) Fry
one to two pieces of chicken at a time until light golden brown, 2-3
minutes, flipping over halfway through. Remove chicken to a plate with
paper towels or clean cotton towels to drain excess oil. Be sure to
allow the hot oil to return to about \unit[370\0]{F.} in between
batches to ensure crispiness of the finished product.
\end{recipe}
