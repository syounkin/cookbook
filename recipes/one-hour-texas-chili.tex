\begin{recipe}{One-Hour Texas Chili (sgy)}{}{}
\freeform \url{http://homesicktexan.blogspot.com/2010/05/frito-pie-with-texas-chili.html}
\ing[6]{}{Guajillo chiles}
\ing[2]{}{Chipotle chiles}
Remove stems and seeds from chiles.  In a large skillet, preferably cast iron, heat the Guajillo and Chipotle dried chiles on medium-high heat about a minute on each side. Turn off the heat, fill the skillet with water and let the chiles soak until rehydrated, about half an hour.
\ing[1]{}{onion, diced}
\ing[4]{cloves}{garlic, minced}
\ing[1]{tbsp.}{vegetable oil or bacon grease}
In a large pot or Dutch oven cook the onions until translucent, about 10 minutes, in vegetable oil or bacon grease. Throw in the garlic and cook for another minute. Place cooked onion and garlic into a blender.
\ing[2]{lbs.}{beef, coarsely ground}
Brown the beef in the pot used for the onions and garlic, stirring occasionally, until lightly browned on each side, about 10 minutes.
\ing[4]{}{Dundicut chiles}
\ing[1]{tbsp.}{cumin}
\ing[1]{tsp.}{oregano}
\ing[\fr12]{tsp.}{cloves, ground}
\ing[\fr12]{tsp.}{cinnamon, ground}
\ing[1]{c.}{water}
Drain the chiles from the soaking water and add them to the blender along with the Dundicut chiles. Add the cumin, oregano, clove, cinnamon and water. Blend until smooth.
\ing[4]{c.}{beer or water}
Add the chile puree and enough water, or beer, to cover the meat, heat on high until boiling, and then simmer uncovered on low heat for 45 minutes, stirring occasionally.
\ing[2]{tsp.}{corn meal or masa harina}
\ing{}{cumin}
\ing{}{salt \& black pepper}
\ing{}{juice of one lime}
After 45 minutes, add salt, black pepper and more cumin to taste. Also, if the chili isn't thick enough for your taste, slowly stir in the masa harina.  Add the lime juice and then cook for 15 more minutes.
\end{recipe}
%% \begin{recipe}{One-Hour Texas chili}{}{}
%% \freeform \url{http://homesicktexan.blogspot.com/2010/05/frito-pie-with-texas-chili.html}
%% \ing[6]{}{ancho chiles, stems and seeds removed}
%% \ing[2]{}{morita chiles, stems and seeds removed}
%% In a large skillet, preferably cast iron, heat the ancho and morita dried chiles on medium-high heat about a minute on each side. Turn off the heat, fill the skillet with water and let the chiles soak until rehydrated, about half an hour.
%% \ing[1]{}{onion, diced}
%% \ing[4]{cloves}{garlic, minced}
%% \ing[1]{tbsp.}{vegetable oil or bacon grease}
%% In a large pot or Dutch oven, in 1 tablespoon of vegetable oil, cook the onions until translucent, about 10 minutes. Throw in the garlic and cook for another minute. Place cooked onion and garlic into a blender.
%% \ing[2]{lbs.}{beef, coarsely ground}
%% Form the ground beef into little balls, about the size of a \fr12-inch marble (This does not need to be perfect, so don't spend too much time doing this. The purpose is to emulate chili chuck, a very coarse grind of beef sold in Texas). In the same large pot, heat the meat, while stirring occasionally, until lightly browned on each side, about 10 minutes. 
%% \ing[4]{}{pequin chiles}
%% \ing[1]{tbsp.}{cumin plus more to taste}
%% \ing[1]{tsp.}{oregano}
%% \ing[\fr12]{tsp.}{ground clove}
%% \ing[\fr12]{tsp.}{ground cinnamon}
%% Drain the chiles from the soaking water and add them to the blender along with the chile pequin (you don’t need to pre-soak these little chilis). Add the cumin, oregano, clove, cinnamon and one cup of water. Blend until smooth.
%% \newstep Add the chile puree and enough water (or beer, if you prefer) to cover the meat (about four cups), heat on high until boiling and then simmer uncovered on low heat for 45 minutes, stirring occasionally.
%% \ing{}{salt \& black pepper}
%% \ing[2]{tsp.}{corn meal or masa harina}
%% \ing{}{juice of one lime}
%% After 45 minutes, add salt and black pepper to taste and feel free to add more cumin if you feel the chili needs it. Also, if the chili isn't thick enough for you taste, slowly stir in the masa harina. (optional, but will thicken chili if needed) Add the lime juice and then cook for 15 more minutes.
%% \end{recipe}
