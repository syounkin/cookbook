\index{Frosting!Hough}
\begin{recipe}{Hough Bakery Frosting}{\unit[6]{cups}}{\unit[20]{minutes}}
\freeform Hough Bakery in Cleveland was legendary, and a little piece
of Julie died when it closed in the nineties. Luckily, somehow Laurie
and Bill Buss got their hands on the recipe for Hough Bakery Frosting
and it has been literally the icing on nearly every birthday cake in
Julie's family for several decades. It is light and fluffy, keeps well
at room temperature and doesn't have the same cloying fat taste from
traditional buttercream. Makes spectacular cakes when colored with
food coloring and piped in the shape of flowers, lettering and
anything else you might want.\\

The Buss recipe calls for almond extract but Julie prefers to use
vanilla. A single recipe makes plenty for a standard two-layer 9-inch
round cake and will result in leftover frosting.

\ing[\fr12]{c.}{half \& half}
\ing[1]{tsp}{vanilla extract}
\ing[1]{tsp}{salt}
Combine in a small bowl.
\ing[2]{lbs}{powdered sugar, sifted}
\ing[2]{sticks}{butter}
\ing[1]{c.}{vegetable shortening}

Cream vegetable shortening and butter together in a stand mixer until
smooth. With the mixer on a medium speed (but not fast enough to fling
sugar out of the mixing bowl), add half of the powdered sugar 3
tablespoons at a time. Thereafter alternate 3-tbsp additions of
powdered sugar with a few tablespoons of the half \& half mixture
until all ingredients have been incorporated. Add more or less of the
half \& half mixture to obtain the desired consistency.

\end{recipe}

